\documentclass[12pt, a4paper] {ncc}
\usepackage[utf8] {inputenc}
\usepackage[T2A]{fontenc}
\usepackage[english, russian] {babel}
\usepackage[usenames,dvipsnames]{xcolor}
\usepackage{listings,a4wide,longtable,amsmath,amsfonts,graphicx,tikz}
\usepackage{indentfirst,verbatim,pdfpages}

\lstset{
  basicstyle=\footnotesize,        % the size of the fonts that are used for the
  breakatwhitespace=false,         % sets if automatic breaks should only happen
  breaklines=true,                 % sets automatic line breaking
  frame=single,	                   % adds a frame around the code
  keepspaces=true,                 % keeps spaces in text, useful for keeping
  numbers=left,                    % where to put the line-numbers; possible
  numbersep=5pt,                   % how far the line-numbers are from the code
  numberstyle=\tiny\color{gray},   % the style that is used for the line-numbers
  rulecolor=\color{black},         % if not set, the frame-color may be changed
  showspaces=false,                % show spaces everywhere adding particular
  showstringspaces=false,          % underline spaces within strings only
  showtabs=false                   % show tabs within strings adding particular
}

\usetikzlibrary{shapes,arrows}

\begin{document}
\setcounter{figure}{0}
\frenchspacing
\pagestyle{empty}
% ============================ ТИТУЛЬНЫЙ ЛИСТ ================================
\begin{center}
     Национальный исследовательский университет информационных технологий,
                              механики и оптики.\\
                       Кафедра вычислительной техники.\\
                Системы ввода-вывода и периферийные устройства.
\end{center}
\vspace{\stretch{2}}
\begin{center}
                         {\bf Лабораторная работа №3, этап 1}\\
          Разработка синтезируемого описания микропроцессорной системы
                                {\sl 7 вариант}
\end{center}
\vspace{\stretch{3}}
\begin{flushright}
                                        Работу выполнили студенты группы P3415\\
                                                            {\it Фомин Евгений},
                                                         {\it Халанский Дмитрий}
\end{flushright}
\vspace{\stretch{4}}
\begin{center}
                                      2016
\end{center}
\newpage
% ======================== КОНЕЦ ТИТУЛЬНОГО ЛИСТА ============================

% ================================ ОТЧЁТ =====================================
\pagestyle{plain}

\tableofcontents

\section{Задание}

Требуется отредактировать проект MIPSFpga таким образом, чтобы он поддерживал
работу с Pmod OLED посредством шины Amba 3 Lite.

\section{Функциональность системы}

Нужно, чтобы со стороны процессора было возможно:

\begin{itemize}
    \item Отправлять по одному байту команды и данные на Pmod OLED через
        интерфейс SPI;
    \item Обращаться к остальным пинам, которые предоставляются устройством.
\end{itemize}

\section{Описание подсистем}

\subsection{Контроллер ввода-вывода SPI}

\subsubsection{Функционал}

\paragraph{Отправка байта} Отправка одного байта по интерфейсу SPI.

\paragraph{Проверка статуса контроллера} Возврат слова, которое включает:

\begin{itemize}
	\item Текущее состояние контроллера: ожидание/отправка;
	\item Байт, полученный в рамках последней завершённой транзакции.
\end{itemize}

\subsubsection{Регистровая карта}

\begin{tabular}{|c|c|l|}
        \hline
        \bf Адрес & \bf Бит & \bf Описание \\
        \hline
        \tt 0x0000 & \tt [7:0] & Байт, который надо отослать \\
        \hline
        \tt 0x0004 & \tt [7:0] & Последний полученный байт \\
                   & \tt [8] & 1~--- если идёт отправка \\
        \hline
\end{tabular}

\subsection{Контроллер дискретного ввода-вывода}

\subsubsection{Функционал}

\paragraph{Установка состояний регистров} Доступно 8 16-битных регистров,
состояние которых возможно изменить путём записи в них. При этом верхняя часть
отправляемого слова представляет собой маску, которую надо применить при записи:
значение бита обновляется, если соответствующий ему бит в маске установлен.

\paragraph{Считывание состояний регистров} Также доступно дополнительные 8
16-битных регистров, доступных только для чтения. Таким образом, всего можно
читать значения 32 регистров.

\subsubsection{Регистровая карта}

\begin{tabular}{|c|c|l|}
        \hline
        \bf Адрес & \bf Бит & \bf Описание \\
        \hline
        \tt 0x0000, 0x0004... 0x1C & \tt [15:0] & Данные \\
                                   & \tt [31:16] & Маска для записи \\
        \hline
        \tt 0x0020, 0x0024... 0x3C & \tt [15:0] & Данные \\
        \hline
\end{tabular}        

\section{Временные диаграммы}

\includegraphics[width=0.95\textwidth]{scrot}

\section{Исходные тексты}

\lstinputlisting[language=Verilog]{verilog/spi.v}
\lstinputlisting[language=Verilog]{verilog/spi_amba_connector.v}
\lstinputlisting[language=Verilog]{verilog/spi_controller.v}
\lstinputlisting[language=Verilog]{verilog/discrete.v}

% ============================ КОНЕЦ ОТЧЁТА ==================================

\end{document}
