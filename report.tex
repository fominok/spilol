\documentclass[12pt, a4paper] {ncc}
\usepackage[utf8] {inputenc}
\usepackage[T2A]{fontenc}
\usepackage[english, russian] {babel}
\usepackage[usenames,dvipsnames]{xcolor}
\usepackage{listings,a4wide,longtable,amsmath,amsfonts,graphicx,tikz}
\usepackage{indentfirst,verbatim,pdfpages}

\usetikzlibrary{shapes,arrows}

\begin{document}
\setcounter{figure}{0}
\frenchspacing
\pagestyle{empty}
% ============================ ТИТУЛЬНЫЙ ЛИСТ ================================
\begin{center}
     Национальный исследовательский университет информационных технологий,
                              механики и оптики.\\
                       Кафедра вычислительной техники.\\
                Системы ввода-вывода и периферийные устройства.
\end{center}
\vspace{\stretch{2}}
\begin{center}
                         {\bf Лабораторная работа №2}\\
          Разработка высокоуровневой модели микропроцессорной системы
                                {\sl 7 вариант}
\end{center}
\vspace{\stretch{3}}
\begin{flushright}
                                        Работу выполнили студенты группы P3415\\
                                                            {\it Фомин Евгений},
                                                         {\it Халанский Дмитрий}
\end{flushright}
\vspace{\stretch{4}}
\begin{center}
                                      2016
\end{center}
\newpage
% ======================== КОНЕЦ ТИТУЛЬНОГО ЛИСТА ============================

% ================================ ОТЧЁТ =====================================

\tableofcontents

\section{Задание}

\subsection{Системная часть}

Требуется написать высокоуровневую модель системы, состоящей из процессора,
нескольких периферийных устройств и инфраструктуры для их сообщения.

В качестве периферийного устройства, определяемого вариантом, выбран ЖК-экран
PmodOLED.

Его возможности включают в себя:

\begin{itemize}
        \item Вывод монохромного изображения на экран разрешением $128 \times
                32$;
        \item Настройка яркости;
        \item Инвертирование изображения;
        \item Динамическая прокрутка изображения по горизонтали и вертикали.
\end{itemize}

\section{Функциональность системы}

Система должна представлять программному обеспечению следующие возможности по
управлению дисплеем:

\begin{itemize}
        \item Установка пикселей как по одному, так и пачкой по 32;
        \item Задание яркости экрана;
        \item Включение и выключение экрана;
        \item Конфигурация автоматической прокрутки экрана.
\end{itemize}

Также нужно поддерживать такие возможности дополнительной инфраструктуры:

\begin{itemize}
        \item Считывание состояния переключателей;
        \item Установка состояния светодиодов.
\end{itemize}

\subsection{Прикладная часть}

Для демонстрации функциональности системы в неё встроено прикладное ПО, которое
выполняет такие задачи:

\begin{itemize}
        \item Диод горит, если соответствующий ему движковый переключатель в
                верхнем положении;
        \item Младший движковый переключатель включает и выключает дисплей;
        \item Пять следующих движковых переключателей определяют символ в
                диапазоне \texttt{[A-Z].,?!-}, причём состоянию, при котором ни
                один переключатель не в верхнем положении, соответствует
                отсутствие символа. Выбранный символ отображается на дисплее;
        \item Следующие четыре переключателя задают один из 16 уровней яркости
                дисплея;
        \item Следующий переключатель задаёт, инвертированное ли изображение
                поступает на экран;
        \item Следующие два переключателя задают скорость горизонтальной
                прокрутки изображения;
        \item Ещё два переключателя определяют скорость вертикальной прокрутки.
\end{itemize}

\section{Модель}

\subsection{Контроллер ввода-вывода для экрана}

\subsubsection{Функционал}

\paragraph{Установка пикселей} Возможно установить значение одного пикселя. Для
этого нужно указать его линейный адрес, который определяется как $\textit{номер
столбца} \cdot 32 + \textit{номер строки}$. Номер строки имеет диапазон $[0;
32)$, номер столбца~--- $[0; 127)$.

\paragraph{Установка столбов} Можно установить столбец целиком. Для этого нужно
указать его адрес, определяемый как $\textit{номер столбца} \cdot 4$, и
передать 32 бита, определяющих пиксели в столбце.

\paragraph{Инверсия} Пиксели можно отображать прямым и обратным образом. При
этом механизм установки пикселей не учитывает текущего метода отображения.
Таким образом, если в прямом режиме команда установки пикселя приводит к его
окрашиванию, то в обратном она приводит к его снятию.

\paragraph{Переворачивание изображения} Контроллер позволяет перевернуть
изображение по вертикали. Возможно, это полезно.

\paragraph{Прокрутка} Прокрутка задаётся двумя числами: скоростью
горизонтальной прокрутки и вертикальной составляющей прокрутки. Вертикальная
составляющая~--- это количество пикселей, на которое в каждый такт прокрутки
изображение сдвигается вертикально. Скорость задаётся тремя битами,
вертикальная составляющая~--- шестью.

\subsubsection{Регистровая карта}

\begin{tabular}{|c|c|l|}
        \hline
        \bf Адрес & \bf Бит & \bf Описание \\
        \hline
        \tt 0x0000-0x0FFF & \tt [0] & Один пиксель \\
        \hline
        \tt 0x1000-0x11FC & \tt [31:0] & Столбец пикселей \\
        \hline
        \tt 0x2000 & \tt [0] & Включен ли экран \\
        \hline
        \tt 0x2004 & \tt [0] & Инвертированное ли изображение \\
                   & \tt [1] & Перевёрнутое ли изображение \\
                   & \tt [5:2] & Яркость экрана от 0 до 16 \\
                   & \tt [8:6] & Скорость прокрутки \\
                   & \tt [14:9] & Вертикальная составляющая прокрутки \\
        \hline
\end{tabular}

\subsection{Контроллер дискретного ввода-вывода}

\subsubsection{Функционал}

\paragraph{Установка и считывания состояния диодов} Контроллер дискретного
ввода-вывода позволяет задать, какие диоды горят и какие нет, а также
определить их текущее состояние. Считываются и устанавливаются состояния всех
диодов разом.

\paragraph{Считывание состояния движковых переключателей} Пользователь может
сменить состояние переключателя. Считываются состояния всех переключателей
разом.

\subsubsection{Регистровая карта}

\begin{tabular}{|c|c|l|}
        \hline
        \bf Адрес & \bf Бит & \bf Описание \\
        \hline
        \tt 0x0000 & \tt [15:0] & Состояние диодов \\
        \hline
        \tt 0x0004 & \tt [15:0] & Состояние движковых переключателей \\
        \hline
\end{tabular}

\subsection{Коммутатор шины}

На коммутаторе шины располагаются контроллер периферийного устройства и
контроллер дискретного ввода-вывода.

Пока доступны только два ведомых устройства, их адреса определяются двумя
младшими битами старших двух байтов следующим образом:

\begin{tabular}{|c|c|l|}
        \hline
        $a_{17}$ & $a_{16}$ & Ведомое устроство \\
        \hline
        \tt 0 & \tt 1 & Контроллер ввода-вывода периферийного устройства \\
        \hline
        \tt 1 & \tt 0 & Контроллер дискретного ввода-вывода \\
        \hline
        \tt 0 & \tt 0 & Ничего \\
        \cline{1-2}
        \tt 1 & \tt 1 & \\
        \hline
\end{tabular}

Для добавления новых устройств требуется:

\begin{enumerate}
        \item Изменить демультиплексор \texttt{HSEL}, чтобы добавить новый
                сигнал;
        \item Добавить ещё одну шину и добавить её в мультиплексор
                \texttt{HRDATA}.
\end{enumerate}

\subsection{Периферийное устройство}

ЖК-экран не предоставляет по шине SPI никаких данных, только принимает
сторонние. Поэтому в качестве периферийного устройства работает отдельная
программа, которая считывает из сокета данные и демонстрирует текущее состояние
буфера кадров. Модуль, отвечающий за периферийное устройство, отсылает команды
в сокет.

\subsection{Программное обеспечение}

Программное обеспечение представлено пользовательской программой и набором
драйверов.

\begin{description}
        \item[Драйвер шины] Реализует чтение и запись четырёх байтов по
                заданному адресу.
        \item[Драйвер контроллера дискретного ввода-вывода] Позволяет получить
                состояния диодов и переключателей, а также установить значения
                как всех диодов, так и одного из них.
        \item[Драйвер контроллера экрана] Позволяет отослать произвольную
                битовую карту размерностью не более $128 \times 32$ на дисплей,
                инвертировать картинку, установить яркость экрана, а также
                свойства прокрутки.
        \item[Адаптеры] Адаптеры позволяют обращаться к контроллерам и вызывать
                их функции, несмотря на то, что они находятся за шиной.
\end{description}

Программное обеспечение вызывает функции, предоставляемые драйверами, для
достижения конкретного результата.

\begin{figure}[h!]
        \centering
        \tikzstyle{decision} = [draw, diamond, aspect=2]
        \tikzstyle{block} = [draw, rectangle]
        \begin{tikzpicture}[node distance=4em, auto]

                \node[draw, ellipse] (begin) {Begin};
                \node[below of=begin, node distance=2em] (beginst) {};
                \node[right of=beginst, xshift=16em]
                        (beginst2) {};
                \node[block, below of=beginst, node distance=2em] (init0)
                        {$d \leftarrow \texttt{turn\_get\_state()}$};
                \node[decision, below of=init0] (LEDen) {$d_0$};
                \node[block, below of=LEDen, xshift=12em] (LEDdoon)
                        {\texttt{led\_enable()}};
                \node[block, below of=LEDen, xshift=-12em] (LEDdooff)
                        {\texttt{led\_disable()}};
                \node[below of=LEDen, node distance=8em] (LEDenaft) {};
                \node[block, below of=LEDenaft] (LEDsym)
                        {$\texttt{led\_display}
                        \left(\texttt{chars}_{\frac 1 2 \left(d
                        \land \mathtt{0x111110}\right)}\right)$};
                \node[block, below of=LEDsym] (LEDbr)
                        {$\texttt{led\_set\_brightness}
                        \left(\frac 1 {2^6} \left(d
                        \land \mathtt{0x1111000000}\right)\right)$};
                \node[block, below of=LEDbr] (LEDinv)
                        {$\texttt{led\_invert}\left(d_{10}\right)$};
                \node[block, below of=LEDinv] (LEDmirror)
                        {$\texttt{led\_mirror}\left(d_{11}\right)$};
                \node[block, below of=LEDmirror] (LEDspeed)
                        {$\texttt{led\_set\_scroll\_speed}\left(
                        2 \cdot d_{13} + d_{12}\right)$};
                \node[block, below of=LEDspeed] (LEDvspeed)
                        {$\texttt{led\_set\_vert\_speed}\left(
                        2 \cdot d_{15} + d_{14}\right)$};
                \node[block, below of=LEDvspeed] (diods)
                        {$\texttt{diods\_set}\left(d\right)$};
                \node[below of=diods, node distance=2em] (end) {};

                \draw[->] (begin) -- (init0);
                \draw[->] (init0) -- (LEDen);
                \draw[->] (LEDen) -| node[near start]{yes} (LEDdoon);
                \draw[->] (LEDen) -| node[near start]{no} (LEDdooff);
                \draw[  ] (LEDdoon) |- (LEDenaft.center);
                \draw[  ] (LEDdooff) |- (LEDenaft.center);
                \draw[->] (LEDenaft.center) -- (LEDsym);
                \draw[->] (LEDsym) -- (LEDbr);
                \draw[->] (LEDbr) -- (LEDinv);
                \draw[->] (LEDinv) -- (LEDmirror);
                \draw[->] (LEDmirror) -- (LEDspeed);
                \draw[->] (LEDspeed) -- (LEDvspeed);
                \draw[->] (LEDvspeed) -- (diods);
                \draw[  ] (diods) -- (end.center);
                \draw[  ] (end.center) -| (beginst2.center);
                \draw[->] (beginst2.center) -- (beginst.center);

        \end{tikzpicture}
        \caption{Блок-схема прикладной программы}
\end{figure}

% ============================ КОНЕЦ ОТЧЁТА ==================================

\end{document}
